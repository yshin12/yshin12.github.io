%2multibyte Version: 5.50.0.2953 CodePage: 1252
%\input{tcilatex}
%\theoremstyle {assumption}


\documentclass[beamer, t]{beamer}
%%%%%%%%%%%%%%%%%%%%%%%%%%%%%%%%%%%%%%%%%%%%%%%%%%%%%%%%%%%%%%%%%%%%%%%%%%%%%%%%%%%%%%%%%%%%%%%%%%%%%%%%%%%%%%%%%%%%%%%%%%%%%%%%%%%%%%%%%%%%%%%%%%%%%%%%%%%%%%%%%%%%%%%%%%%%%%%%%%%%%%%%%%%%%%%%%%%%%%%%%%%%%%%%%%%%%%%%%%%%%%%%%%%%%%%%%%%%%%%%%%%%%%%%%%%%
\usepackage{amsfonts}
\usepackage{amssymb}
\usepackage{color}
\usepackage{amsmath}
\usepackage{colortbl}
\usepackage{graphicx}
\usepackage{hyperref}
\usepackage{pgfpages}

\usepackage{booktabs}
\usepackage{tikz}
\usepackage{multirow}
\usepackage{dsfont}
\usepackage{comment}
\usepackage{threeparttable}
\usepackage{ragged2e}
\usepackage{etoolbox}
\usepackage{lipsum}

\usetikzlibrary{calc,arrows}

\usepackage[ruled]{algorithm2e}
\SetKwInput{KwInput}{Input}                % Set the Input
\SetKwInput{KwOutput}{Output}              % set the Output
%\usepackage[dvipsnames]{xcolor}
%\apptocmd{\frame}{}{\justifying}{} % Allow optional arguments after frame.


%\usepackage{siunitx}

%\usepackage{supertabular}

%\setbeameroption{hide notes} % Only slides
%\setbeameroption{show only notes} % Only notes
%\setbeameroption{show notes on second screen=right} % Both


\setcounter{MaxMatrixCols}{10}
%TCIDATA{OutputFilter=LATEX.DLL}
%TCIDATA{Version=5.50.0.2953}
%TCIDATA{Codepage=1252}
%TCIDATA{<META NAME="SaveForMode" CONTENT="1">}
%TCIDATA{BibliographyScheme=Manual}
%TCIDATA{Created=Tuesday, October 31, 2017 14:48:15}
%TCIDATA{LastRevised=Thursday, November 02, 2017 13:43:02}
%TCIDATA{<META NAME="GraphicsSave" CONTENT="32">}
%TCIDATA{<META NAME="DocumentShell" CONTENT="My Style\beamer_simple">}
%TCIDATA{CSTFile=beamer.cst}

\newcommand{\cref}[2][1]{{\textup{(\hyperref[#2]{\ref*{#2}$_{#1}$})}}}
\newcommand{\eq}[1]{\begin{align}#1\end{align}}
\newcommand{\eqs}[1]{\begin{align*}#1\end{align*}}
\newcommand{\tcr}{\textcolor{red}}
\newcommand{\eps}[0]{\ensuremath{\varepsilon}}
\newcommand{\dt}{\delta}
\newcommand{\what}{\widehat}
\newcommand{\ap}{\alpha}
\newcommand{\bt}{\beta}
\newcommand{\ld}{\lambda}
\newcommand{\gm}{\gamma}
\newcommand{\sgn}{\mathrm{sgn}}

\newcommand{\bsk}{\bigskip}
\newcommand{\lt}{\left}
\newcommand{\rt}{\right}
\newcommand{\rarrow}{\rightarrow}
\newcommand{\bit}{\begin{itemize}}
\newcommand{\eit}{\end{itemize}}
\newcommand{\bft}{\mathbf{t}}
\newcommand\SPi{\mathrm{\Pi}}


\newcommand{\bone}{\mbox{\bf 1}}
\newcommand{\bsone}{\mbox{\scriptsize \bf 1}}
\newcommand{\bzero}{\mbox{\bf 0}}
\newcommand{\bveps}{\mbox{\boldmath $\varepsilon$}}
\newcommand{\bet}{\mbox{\boldmath $\eta$}}
\newcommand{\bxi}{\mbox{\boldmath $\xi$}}
\newcommand{\beps}{\mbox{\boldmath $\varepsilon$}}
\newcommand{\bmu}{\mbox{\boldmath $\mu$}}
\newcommand{\bgamma}{\mbox{\boldmath $\gamma$}}
\newcommand{\mv}{\mbox{V}}
\newcommand{\bSigma}{\mbox{\boldmath $\Sigma$}}
\newcommand{\bOmega}{\mbox{\boldmath $\Omega$}}
\newcommand{\norm}{\bigg{|}}
\newcommand{\bPhi}{\mbox{\boldmath $\Phi$}}
\newcommand{\hB}{\widehat \bB}
\newcommand{\ty}{\widetilde \by}
\newcommand{\tf}{\widetilde \bff}
\newcommand{\tB}{\widetilde \bB}
\newcommand{\tb}{\widetilde \bb}
\newcommand{\tA}{\widetilde A}
\newcommand{\hb}{\widehat \bb}
\newcommand{\hE}{\widehat \bE}
\newcommand{\hF}{\widehat \bF}
\newcommand{\halpha}{\widehat \alpha}
\newcommand{\hu}{\widehat \bu}
\newcommand{\hvar}{\widehat \var}
\newcommand{\hcov}{\widehat \cov}
\newcommand{\hbveps}{\widehat\bveps}
\newcommand{\hSig}{\widehat\Sig}
\newcommand{\hsig}{\widehat\sigma}
\newcommand{\hmu}{\widehat\bmu}
\newcommand{\htau}{\widehat\tau}
\newcommand{\hxi}{\widehat\bxi}
\newcommand{\heq}{\ \widehat=\ }
\newcommand{\sam}{_{\text{sam}}}
\newcommand{\cov}{\mathrm{cov}}
\newcommand{\Sig}{\mathbf{\Sigma}}
\newcommand{\veps}{\varepsilon}
\newcommand{\tr}{\mathrm{tr}}
\newcommand{\tcb}{\textcolor{blue}}
\newcommand{\diag}{\mathrm{diag}}
\newcommand{\vecc}{\mathrm{vec}}
\newcommand{\bw}{\mbox{\bf w}}
\newcommand{\var}{\mathrm{var}}
\newcommand{\beq}{\begin{eqnarray*}}
\newcommand{\eeq}{\end{eqnarray*}}
\newcommand{\fm}[1]{\begin{frame}#1\end{frame}}
\newcommand{\fmt}{\frametitle}
\newcommand{\fmst}{\framesubtitle}
\DeclareMathOperator*{\argmin}{arg\,min}
\DeclareMathOperator*{\argmax}{arg\,max}


% \DeclareGraphicsExtensions{.eps,.bmp,.png,.jpg}
% \DeclareGraphicsRule{.png}{bmp}{}{}



\setbeamertemplate{itemize items}[ball]
\setbeamertemplate{itemize subitem}[ball]
\setbeamertemplate{itemize subsubitem}[ball]

\setbeamertemplate{navigation symbols}{}

\addtobeamertemplate{navigation symbols}{}{%
    \usebeamerfont{footline}%
    \usebeamercolor[fg]{footline}%
    \hspace{1em}%
    \insertframenumber/\inserttotalframenumber
}

\useoutertheme[footline=empty,subsection=false]{miniframes}
\useinnertheme{circles}

%\AtBeginSection[]{
%  \begin{frame}
%  \vfill
%  \centering
%  \begin{beamercolorbox}[sep=30pt,center,shadow=true,rounded=true]{title}
%    \usebeamerfont{title}\insertsectionhead\par%
%  \end{beamercolorbox}
%  \vfill
%  \end{frame}
%}


\newtheorem{thm}{Theorem}[section]
\newtheorem{defn}{Definition}[section]
\newtheorem{lem}{Lemma}[section]
\newtheorem{cor}{Corollary}[section]
\newtheorem{assum}{Assumption}
%\newenvironment{stepenumerate}{\begin{enumerate}[<+->]}{\end{enumerate}}
%\newenvironment{stepitemize}{\begin{itemize}[<+->]}{\end{itemize} }
%\newenvironment{itemize}{\begin{enumerate}[<+-| alert@+>]}{\end{enumerate}}
%\newenvironment{itemize}{\begin{itemize}[<+-| alert@+>]}{\end{itemize} }
%\institute{\inst {1}McMaster University}
%\input{tcilatex}
\begin{document}

\title{Fast Inference for Quantile Regression \\ 
with Tens of Millions of Observations}
\author[Lee, Liao, Seo \& Shin]{Sokbae Lee, Yuan Liao, Myung Hwan Seo \& Youngki Shin \\
\vskip5pt
\tiny Columbia, Rutgers, SNU, \& McMaster}
\date{\\
University of California, Riverside\\
December 6, 2022}
\maketitle






% Outline frame
% \begin{frame}{Outline}     \tableofcontents \end{frame}

\section{Introduction}


\begin{frame}{Main Object}
In this paper, we tackle on the \tcb{inference} problem of quantile regression (QR) with $(n,d)\sim(10^7,10^3)$:
		$$
		y_i = x_i'\beta^*+ \varepsilon_i,\quad P(\varepsilon\leq 0|x_i)= \tau.
		$$

	\begin{itemize}

		\item We estimate the wage structure (college premium) using the data from IPUMS USA. The sample size of each year is over 14 millions. 
		
		\item We also apply many controls to mitigate the bias, which turns out to be over 1,000. 
		
		\item For a smaller sample size, we need additional assumptions, e.g.\ sparsity in the lasso. 

		
	\end{itemize}
	
\end{frame}

\begin{frame}
\frametitle{Big Picture}
\bit
	\item Standard asymptotics: $n \gg d$. E.g.~$n=1000$ and $d=20$.
	\item High-dimensional approach: $n \sim d$ or $n \ll p$. E.g.~$n \sim \exp(d)$
	\bit
		\item[-] Threshold regression: LSS (2016, JRSSB), LLSS (2018, JASA), LLSS (2021a, AOS).
		\item[-] Filtering and prediction: LLSS (2021b, JOE), LS (forthcoming, JOE)
	\eit
	\item The current problem is $n=10^7$ and $d=10^3$. Can we go back to the standard framework?

	\item It turns out that we need \tcb{a novel approach} because of the implementation issue.
\eit
\end{frame}

\begin{frame}{Standard QR Estimator}
	Let $\{Y_i\equiv((y_i,x_i) \in \mathbb{R}^(1+d): i=1,\ldots,n\}$ be a random sample generated from 
		$y_i = x_i'\beta^*+ \varepsilon_i,\quad P(\varepsilon\leq 0|x_i)= \tau.$
	\bigskip 

	The object of interest is
	$$
		\beta^* := \arg\min_{\beta \in \mathbb{R}^d} Q(\beta),
	$$
	where 
	\begin{align*}
	Q(\beta) & := \mathbb{E}[q(\beta,Y_i)]\\
	q(\beta, Y_i) &:=(y_i-x_i'\beta)(\tau-I\{y_i -x_i'\beta \le0\}).
	\end{align*}

	The QR estimator is defined as
	$$
		\widehat{\beta}_n := \arg\min_{\beta \in \mathbb{R}^d} \frac{1}{n} \sum_{i=1}^{n} q(\beta_,Y_i).
	$$
\end{frame}

\begin{frame}{Standard QR Estimator (cont.)}
	The standard M-estimator theory gives us 
	\eqs{
	\sqrt{n}(\widehat\beta_n-\beta^*) \overset{d}{\to} N(0, \tau(1-\tau)H^{-1}\mathbb E[x_ix_i'] H^{-1}),
	}
	where $H= \mathbb E [f_{\varepsilon}(0|x_i)x_ix_i']$ and $f_{\varepsilon}(\cdot|x_i)$ is the conditional distribution of $\varepsilon_i$ given $x_i$

	\bit
		\item Point estimator: Linear programming through interior-point algorithms or smoothing type estimators
		\item Covariance estimator: 
		The \texttt{conquer} method in He et al.~(2021) has received attention and boast its capability to make inference with $ (n,p)=(4000,100) $.
	\eit
	New approach: we propose a stochastic subgradient descent (\tcb{S-subGD}) method with a \tcb{random scaling}.
\end{frame}




\begin{frame}{Gradient Descent}
Let $\beta^*$ be the parameter of interest:
\eqs{
    \beta^{*}:=\arg\min_{\beta\in\mathbb{R}^{d}}Q\left(\beta\right)
}
where $Q:=E[q(\beta,Y)]$ and $q$ is diff.~and convex. Let $\left\{ Y_{t}\right\} _{t=1}^{n}$ be a random sample. The sample analogue of the FOC is
\eqs{
    \frac{1}{n}\sum_{t=1}^n \nabla q\left(\hat{\beta},Y_{t}\right) = 0.
}

If we don't have a reduced form solution, we can solve it iteratively:
\eqs{
    \hat{\beta}_{m} = \hat{\beta}_{m-1} - \gamma_m \frac{1}{n}\sum_{t=1}^n \nabla q\left(\hat{\beta}_{m-1},Y_{t}\right).
}
\end{frame}


\begin{frame}{Stochastic Gradient Descent}
Limitations of gradient descent:
\bit
    \item It calculates the derivatives for the entire dataset.
    \item It requires a larger memory size as the dataset increases.
\eit
Binding time budget or the memory size occurs more often in modern empirical applications. 
\bigskip

Robbins and Monro (1951) proposed the stochastic gradient descent (SGD) solution path as
        \eqs{
            \beta_{t}=\beta_{t-1}-\gamma_{t}\nabla q\left(\beta_{t-1},Y_{t}\right).
        }

SGD has advantages when we face \tcb{a large-scale dataset} or \tcb{online machine learning}.

Examples: Chen and White (2002), Khan, Lan, and Tamer (2021). 
\end{frame}




\begin{frame}{SGD Averaging}
Recall that we aim to develop \tcb{online inference} with SGD. 
\bigskip

We study the classical Polyak-Ruppert averaging estimator (Polyak (1990) and Ruppert(1988)):
$\bar{\beta}_{n}:=n^{-1}\sum_{t=1}^{n}\beta_{t}$.
\medskip

Polyak and Juditsky (1992)
established regularity conditions under which
the averaging estimator $\bar{\beta}_{n}$ is asymptotically normal:
\begin{align*}
\sqrt{n}\left(\bar{\beta}_{n}-\beta^{*}\right)\overset{d}{\to}
\mathcal{N} (0, \Upsilon),
\end{align*}
where the asymptotic variance $\Upsilon$ has a sandwich form $$\Upsilon := H^{-1}S H^{-1},$$ and
$H := \mathbb{E}[\nabla^{2}Q\left(\beta^{*}\right)]$ is the Hessian matrix
and $S := \mathbb{E} \left[\nabla q\left(\beta^{*},Y\right)\nabla q\left(\beta^{*},Y\right)'\right]$ is the score variance.
\end{frame}


\begin{frame}{SGD Averaging in Online Learning}
In online learning, data arrive sequentially.
\bigskip


The  Polyak-Ruppert estimator $\bar{\beta}_{n}$ can be computed recursively
by the updating rule $$\bar{\beta}_{t}=\bar{\beta}_{t-1}\frac{t-1}{t}+\frac{\beta_{t}}{t},$$
which implies that it is well suited to the online setting.

\bigskip

Examples include
\bit
    \item Linear regression (with a large dataset)
    \item Logistic regression
    \item \tcb{Quantile regression (using a subgradient)}:
    		$$
			\nabla q(\beta; x,y)= - x[u-1\{y\leq x'\beta\}],
			$$
\eit
\end{frame}




\begin{frame}{Overview}
    \vfill
    \bit
        \item Introduction
        \item Inference Methods
        \item Theoretical Results
        \item Monte Carlo Experiments
        \item Application
        \item Conclusion
    \eit
    \vfill
\end{frame}




\section{Inference Methods}

\begin{frame}
   \vfill
   \centering
   \LARGE{Inference Methods}
   \vfill
\end{frame}

  

\begin{frame}{Online Inference}

Although the asymptotic normality result by Polyak and Juditsky (1992) was established about three decades ago, it is only past few years that online inference has gained increasing interest in the literature.
\bigskip

It is challenging to estimate the asymptotic variance $\Upsilon$  in an online fashion.
\bigskip


This is because
the naive implementation of estimating it requires storing all data, thereby losing the advantage of online learning.


\end{frame}



\begin{frame}{Method 1: Plug-In}


Chen et al.~(2020) addressed this issue by  estimating   $H$ and $S$ using the online iterated estimator $\beta_t$, and recursively updating   them   whenever a new observation  is available.

\bigskip 
However, the plug-in estimator requires that the Hessian matrix be computed to estimate $H$.

\bigskip
In other words, it is necessary to have strictly more inputs than  the SGD solution paths $\beta_t$ to carry out inference. It is the case even when a t-statistic is computed for each regression coefficient.

\bigskip
In applications, it can be demanding to compute the Hessian matrix and its inverse. 

%\bigskip
%They do not cover the quantile regression.

\end{frame}



\begin{frame}{Method 2: Batch-Means}

%\item As an alternative, Chen et al. (2020)  proposed a \textcolor{blue}{\textit{batch-means}} estimator that avoids separately estimating   $H^{-1}$ or $S$.


This method proposed by Chen et al.~(2020) and Zhu et al.~(2021) directly estimates the variance of the averaged online estimator $\bar{\beta}_{n}$
by dividing $\{\beta_1,\ldots,\beta_n\}$ into batches with increasing
batch size.

\bigskip 
The batch-means estimator is based on the idea that correlations among batches that are far apart decay exponentially fast; therefore, one can use nonparametric empirical covariance to estimate $\Upsilon$.

\bigskip
However, this approach requires the batch size should increase exponentially fast, and it turns out that the performance is not satisfactory. 
\end{frame}

\begin{comment}  
\begin{frame}{Weakly Dependent Batches}
	%\small 
	
	
	
	\begin{itemize}
		\item Zhu et al (2021)  take batches: e.g.,
		\begin{eqnarray*}
			B_1&=& \{\beta_1\}, \quad B_2=\{\beta_1, \beta_2\},  \cr
			B_3&=&\{\beta_3 \}, \quad B_4=\{\beta_3, \beta_4 \}, \cr
			B_5&=&\{\beta_5 \},\quad B_6=\{\beta_5, \beta_6 \},
		\end{eqnarray*}
		
		\item        		In fact, the gap $B_1, B_3, B_5...$ should be much larger, so that they are weakly dependent.  So the ``batch gap" and ``batch size" are tuning parameters. 
		
		\item The ``weakly dependent" batches are required, and the goal is  the average of blockwise sample variance
		$$
		\widehat \Upsilon  \to ^ P \Upsilon.
		$$	
		
		\item Then
		$$
		\sqrt{n}\widehat \Upsilon^{-1/2}(\bar\beta_n-\beta^*) \to^d N(0, I).
		$$
		
	\end{itemize} 
	
	
\end{frame} 	
\end{comment}

\begin{frame}{Method 3: Bootstrap}


Instead of estimating the asymptotic variance, Fang et al. (2018) proposed a  bootstrap procedure  for  online inference.

\bigskip
Specifically, they proposed to use a large number (say, $B$) of randomly perturbed
SGD solution paths: for all $b = 1, \ldots, B$, starting with $\beta_0^{(b)} = \beta_0$ and then iterating
\begin{equation*}
\beta_{t}^{(b)}=\beta_{t-1}^{(b)}-\gamma_{t} \eta_t^{(b)} \nabla q\left(\beta_{t-1}^{(b)},Y_{t}\right),
\label{eq:SGD1:bt}
\end{equation*}
where $\eta_t^{(b)} > 0$ is an independent and identically distributed random variable that has mean one and variance one.


\bigskip
The bootstrap procedure needs strictly more inputs than computing $\bar{\beta}_{n}$
and can be time-consuming.

\end{frame}



\begin{frame}{Our Approach: Random Scaling}

\begin{itemize}
	\item Lee, Liao, Seo, Shin (LLSS, 2022) proposed not to estimate the asymptotic variance $\Upsilon$,  but to studentize   $\sqrt{n}\left(\bar{\beta}_{n}-\beta^{*}\right)$ via $ \widehat{V}_{n}^{1/2} $, where
	\begin{align*}\label{def:random-scaling}
	\widehat{V}_{n} := \frac{1}{n}\sum_{s=1}^{n}
	\left\{ \frac{1}{\sqrt{n}} \sum_{t=1}^{s} \left( \beta_{t}-\bar{\beta}_{n} \right) \right \}
	\left\{ \frac{1}{\sqrt{n}} \sum_{t=1}^{s} \left( \beta_{t}-\bar{\beta}_{n} \right) \right \}'.
	\end{align*}
	\note{this is the integral of the square of a centered partial sum process or that of empirical Brownian Bridge}
	
	\item
	It converges in distribution to a pivotal distribution up to an unknown scale, which is the same as the asymptotic variance of the average SGD estimator. 
	This leverages insights from the time series literature (e.g.~Kiefer et al.~(2000)).
	
	\item It has been already being adopted in other machine-learning literature like in federated learning (Li et al. 2022), Kiefer-Wolfowitz method (Chen et al. 2021) among others.
	
	
	
	
\end{itemize}


% While the batch-means estimator aims to mitigate the effect of dependence among the sequence of SGD estimates, we \tcb{embrace the dependence} and build a test statistic via random scaling.



\end{frame}



\begin{frame}{Algorithm}

\begin{algorithm}[H]
\footnotesize
 \SetKwInput{KwInit}{Initialize}
 \SetKwInput{KwReceive}{Receive}

 \KwInput{function $\nabla q(\cdot)$, parameters $(\gamma_0, a)$ for step size $\gamma_t=\gamma_0 t^{-a}$ for $t \geq 1$}

 \KwInit{
  set initial values for $\beta_0, \bar{\beta}_0, A_0, b_0$
 }
 \For{$t = 1,2, \ldots $} {
    \KwReceive{new observation $Y_t$}
    $\beta_{t}=\beta_{t-1}-\gamma_{t}\nabla q\left(\beta_{t-1},Y_{t}\right)$

    $\bar{\beta}_{t}=\bar{\beta}_{t-1}\frac{t-1}{t}+\frac{\beta_{t}}{t}$

    $A_{t} = A_{t-1} + t^{2}\bar{\beta}_{t}\bar{\beta}_{t}'$

    $b_{t} = b_{t-1} + t^{2}\bar{\beta}_{t}$
    
    $c_t =c_{t-1} + t^2$

    Obtain $\widehat{V}_{t}$ by
      \[
      \widehat{V}_{t}
      = t^{-2} \left( A_t - \bar{\beta}_{t} b_{t}' - b_{t}\bar{\beta}_{t}' +\bar{\beta}_{t}\bar{\beta}_{t}'c_t \right)
      \]

    \KwOutput{$\bar{\beta}_t$, $\widehat{V}_{t}$}
 }
\end{algorithm}

\end{frame}


\begin{comment}

\begin{frame}{Criteria for Online Inference Methods}

\begin{table}[htbp]

\caption{Criteria for Online Inference Methods}
\tiny
\begin{center}
\begin{tabular}{lccccc}
\hline
%\\
 & FXY (18) & CLTZ (20) & CLTZ (20) & ZCW (21) & LLSS (22) \\
% \\
Method & Bootstrap & Plug-In  & Batch & Batch & Random \\
 &  &   & Means & Means & Scaling \\
\hline
%\\
\multicolumn{5}{l}{Is it possible} \\
\; to avoid resampling?  & & \checkmark & \checkmark & \checkmark & \checkmark \\
\; to avoid Hessian? &  \checkmark &  & \checkmark & \checkmark & \checkmark \\
%\; to avoid batches? &  \checkmark & \checkmark & & & \checkmark \\
\; to update recursively? & \checkmark &  \checkmark &  & \checkmark & \checkmark \\
\hline
\end{tabular}
\end{center}
\label{tab:criteria}
\begin{minipage}{\textwidth} % choose width suitably
{Note.
FXY (18), CLTZ (20), and ZCW (21)
 refer to
Fang et al. (2018),
Chen et al. (2020),
and
Zhu et al. (2021),
respectively.
  \par}
\end{minipage}
\end{table}

\end{frame}

\end{comment}



\begin{frame}{Additional Remarks/Features}
	
	\begin{itemize}
		\item 
		We set an initial value $\beta_0$ from He et al (2021). Works well with only 1\% or 5\% of the sample. 
		
		\item We set the step size $ \gamma_{t} $ is set closer to upper bound, $ t^{-0.501} $. 

		\item It does not involve any inverse matrix. We can iterate only a sub-matrix of $\widehat{V}_n$. E.g.~a scalar vs.~a $(1000\times 1000)$ matrix.
	\end{itemize}
	
\end{frame}

\begin{comment}
\begin{frame}{Inference for Sub-vectors}
	
	
	\begin{itemize}
		\item Empirical studies often involve many controls: $\beta^*$ is ``long".
		
		\item But of interest is a sub-vector, containing only 1$\sim$2 elements. 
		
		\item It is straightforward to caster sub-vector inference:
		
		\begin{eqnarray}
			\beta_i&=&  \text{full vector}\\
			\bar\beta_i&=&  \text{sub vector}\\
			\widehat V_i&=&  \text{sub matrix}
		\end{eqnarray}
		
		
	\end{itemize}
	
\end{frame}
\end{comment}


\begin{comment}
\begin{frame}{Forneron and Ng (2020)}
	\begin{itemize}
		\item They proposed 
		$$
		\widehat\beta_i= \widehat\beta_{i-1} -\gamma  \textcolor{blue}{H^{-1}} \nabla q(Y_i, \widehat\beta_{i-1})
		$$
		\item Can consistently estimate the variance using
		$$
		\frac{\gamma^2}{1-(1-\gamma)^2} \frac{1}{n}\sum_i (\widehat\beta_i-\bar\beta_n)(\widehat\beta_i-\bar\beta_n)'
		$$
		where $\bar\beta_n=\frac{1}{n}\sum_i\widehat\beta_i.$
		\item The use of a fixed step size and the Hessian $H^{-1}$ is the key for the consistency: 
		$$
		\widehat\beta_i\approx \rho \widehat\beta_{i-1} + noise 
		$$
		\item Representable as a stationary AR(1) model. 
	\end{itemize}
	
\end{frame}



\begin{frame}{Extensions to cluster-dependent QR}
	Consider 
	$$
	y_{i,t} = x_{i,t}'\beta +\epsilon_{i,t}
	$$
	where $i=1...n$ are ``clusters", and $t=1,...,T_i$ are individuals within clusters.
	
	\begin{itemize}
		\item Allow arbitrary dependence within cluster;  Independence between clusters.
		\item Large $n$, bounded $T_i$.
		\item current theory requires $T_i=T$ for all $i$.
		
		\item A generalization in the spirit of Hansen and Lee (2019, JoE) would be very interesting.
		 
	\end{itemize}
	
\end{frame}


\begin{frame}{Extensions to cluster-dependent QR}
	
	
	\begin{itemize}
		\item Hagemann 2017  proposed a multiplier bootstrap for this setting. 
		
		\item Our proposal: recursively update
		$$
		\beta_i = \beta_{i-1} -\gamma_i \frac{1}{T_i}\sum_{t=1}^{T_i}\nabla q(Y_{i,t}, \beta_i)
		$$
		where $\nabla q(Y, \beta) = -x(\tau -1\{y<x'\beta\})$.
		
		\item Final estimator: $\bar\beta_n$
		
		\item Inference:  Same random-scaling $\widehat V_n$.
	\end{itemize}
	
\end{frame}

\begin{frame}{A random clustering ? }
	
	\begin{enumerate}
		\item 	Consider a random sampling from $W=\left(D_{1},D_{2},X_{1},X_{2},....\right)$. 
		
		\item then sampling scheme for a cluster such that  each cluster $\left(X_{g1},...,X_{g,n_{g}}\right)$
		for $g=1,2,...,G$ is a transformation such that 
		\[
		\left(X_{g1},...,X_{g,n_{g}}\right)=\left(X_{D_{1}},...,X_{D_{2}}\right),
		\]
		where $ n_g =D_2 - D_1 +1 $.
		\item
		Here, we assume $D_{1}\leq D_{2}$ but we do not  assume much
		about other dependence on the vector $W$. Then, 
		\[
		\bar{X}_{g}=n_{g}^{-1}\sum_{i=1}^{n_{g}}X_{g,i}
		\]
		is an iid sequence. That is, $\bar{X}_{g}$ is iid over $g=1,2,...$
		unconditionally, while it is not conditionally on $\left(D_{g,1},D_{g,2}\right)$. 
	\end{enumerate}

\end{frame}

\end{comment}



\section{Theoretical Results}

\begin{frame}
    \vfill
    \centering
    \LARGE{Theoretical Results}
    \vfill
\end{frame}

\begin{frame}{ Functional Central Limit Theorem for Online SGD}

We first  extend
Polyak and Juditsky (1992)'s central limit theorem (CLT) to a \emph{functional} CLT (FCLT) for partial sum process:
\begin{equation*}\label{eq4}
\frac{1}{\sqrt{n}}\sum_{t=1}^{\left[nr\right]}\left(\beta_{t}-\beta^{*}\right)\Rightarrow \Upsilon^{1/2}W\left(r\right),\quad r\in\left[0,1\right],
\end{equation*}
where $\Rightarrow$ stands for the weak convergence in $\ell^{\infty}\left[0,1\right]$
and $W\left(r\right)$ stands for a vector of the independent standard
Wiener processes on $\left[0,1\right]$.

\bigskip
That is, the partial sum of the online updated estimates $\beta_t$ converges weakly to a rescaled Wiener process,

\bigskip 
Note that the scaling $\Upsilon^{1/2}$ is equal to the square root  asymptotic variance of the Polyak-Ruppert average.

\end{frame}

\begin{frame}{PJ's approximation}
	Their CLT is built on a brilliant stochastic approximation that 
	\[  \frac{1}{\sqrt{n}}\sum_{t=1}^{n}\left(\beta_{t}-\beta^{*}\right) = \frac{1}{\sqrt{n}}\sum_{t=1}^{n}H^{-1}\xi_{t} + o_p (1),
	 \]
	 where $ \xi_{t}\ $ is a martingale difference sequence (MDS) whose variance converges to the score variance. 
	 \bigskip
	 
	 For FCLT, it is required to extend the approximation to the uniform approximation. 
\end{frame}


\begin{frame}{Conditions for Quantile Regression}
	Let the partial derivative
	$\frac{d}{de}f_{\varepsilon}(\cdot|x_i)$ exist and assume that 
	\begin{enumerate}
		
		%\item[(i)] for $q(\beta;x, y)= (y-x'\beta)(\tau-I\{y-x'\beta\leq 0\})$,
		% $$\lim\inf_{\left| \beta \right|\to\infty} E [q(\beta; x_i, y_i)] > \mathbb E [q(\beta^*_\tau; x_i, y_i)],$$
		
		\item
		There exist positive constants $\epsilon$ and $c_0$ such that 
		$$
		\inf_{  | \beta-\beta^* |<\epsilon }\lambda_{\min} \left(  \mathbb E [x_ix_i'f_{\varepsilon}(x_i'(\beta-\beta^*)|x_i)] \right) > c_0,
		$$
		
		%	 \item[(i)] For some $\epsilon,c>0$, $\inf_{ | \beta-\beta^* |<\epsilon}\lambda_{\min} (G(\beta))>c$. %$\sup_{| \beta-\beta^* |<\epsilon}\frac{d}{de}f_{\varepsilon}(x_i'(\beta-\beta^*)|x_i)>c $ almost surely and $\mathbb E\|x_i\|^m<C$, where $m=\max\{3, 2/\left(1-a\right)\}$
		
		
		
		\item $\sup_b\mathbb E [ \|x_i\|^3A(b, x_i) ] <C$ for some constant $C < \infty$, where  $ A(b, x_i) :=   \left|\frac{d}{de}f_{\varepsilon}(x_i'b|x_i)\right| + f_{\varepsilon}(x_i'b|x_i),$ % $(x_i, \varepsilon_i)$ is i.i.d. across $t$.
		
		%\item[(iii)]  $\mathbb E[x_ix_i'|\mathcal F_{i-1}]$ does not vary with $t$.
		
		\item $\mathbb E[ (\|x_i\|^6+1)\exp(\|x_i\|^2) ]<C$ for some constant $C < \infty$,
		
		%			\item 			$\mathbb{E}\left\Vert x_{i}\right\Vert ^{2p} < C$ for some constant $C < \infty$ and some $p\geq\left(1-a\right)^{-1}$, where $1/2 < a <1$ is the exponent in the learning rate $\gamma_{i}:=\gamma_0 i^{-a} $. 
	\end{enumerate}
	
	\bigskip
	This is a set of low-level conditions to meet Gadat and Panloup (2022)'s consistency without strong convexity.% and an extension of LLSS (2022)'s FCLT. 
	
	\note{	
		Condition (i) can be viewed as a global identification condition, which is reasonable as $\beta \mapsto q(\beta, Y_i)$ is convex. 
		Conditions (ii)-(iv) impose some moment conditions and are satisfied, for example, if $x_i$ are bounded 
		and 
		both $f_{\varepsilon}(\cdot|x_i)$ and $\frac{d}{de}f_{\varepsilon}(\cdot|x_i)$ are uniformly bounded.
	}
\note{We impose the following moment condition in addition to Polyak and Juditsky (1992)'s assumption:
	
	The mds sequence $\left\{ \xi_{t}:=\nabla Q\left(\beta_{t-1}\right)-\nabla q\left(\beta_{t-1},Y_{t}\right) \right\}_{t\geq1}$ satisfies
	$\mathbb{E}\left\Vert \xi_{t}\right\Vert ^{2p}$
	is bounded, for $p\geq\left(1-a\right)^{-1}$ where $\gamma_{t}=\gamma_0 t^{-a} $ for some $1/2 < a <1$. However, we relax their global regularity condition to local one to include quantile regression.
}
	
\end{frame}


 \begin{frame}{Main Theorem}

Let for any $\ell \leq d$ linear restrictions
\[
H_{0}: R\beta^{*} = c,
\]
where $R$ is  an $(\ell \times d)$-dimensional known matrix of  rank $\ell$
and
$c$ is an $\ell$-dimensional known vector.
\begin{thm}
\label{thm:Wald} Suppose rank$(R)=\ell$. Under the stated  Assumptions and $H_{0}$,
\begin{eqnarray*}
&&n\left(R\bar{\beta}_{n}-c\right)'\left(R\widehat{V}_{n}R'\right)^{-1}\left(R\bar{\beta}_{n}-c\right)\\
&& \overset{d}{\to}W\left(1\right)'\left(\int_{0}^{1}\bar{W}(r)\bar{W}(r)'dr\right)^{-1}W\left(1\right),
\end{eqnarray*}
where $W(\cdot)$ is an $\ell$-dimensional vector of the standard Wiener
processes and $\bar{W}\left(r\right):=W\left(r\right)-rW\left(1\right)$.
\end{thm}

\end{frame}


\begin{frame}{Special Case: t-Statistic}

The t-statistic for each $ j $ converges in distribution:
\begin{align*}\label{t-stat-limit}
\frac{\sqrt{n}\left(\bar{\beta}_{n,j}-\beta_{j}^{*}\right)}{\sqrt{\widehat{V}_{n,jj}}}
\overset{d}{\to}
W_1\left(1\right)  \left[ \int_{0}^{1} \left\{ W_1\left(r\right)-rW_1\left(1\right) \right\}^2 dr\right]^{-1/2},
\end{align*}

 \begin{itemize}
 	\item 
 	The asymptotic distribution is mixed normal and symmetric around zero.
 	
 	\item It  is the same as the distribution of the statistics observed in the estimation of the cointegration vector by Johansen  (1991). They are different statistics but have the identical distribution as functions of the standard Wiener process as shown by Abadir and Paruolo (2002).
 	
 % 	\item Abadir and Paruolo (1997) obtained a closed form density function. 
 \end{itemize}

\end{frame}



\section{Monte Carlo Experiments}

\begin{frame}
	\vfill
	\centering
	\LARGE{Monte Carlo Experiments}
	\vfill
\end{frame}


 
\begin{frame}{Settings} 
	\begin{itemize}
		\item $\mbox{dim}(x_t)=d\in\{10, 30, 180, 320, 1000\}$
		\item $n\in\{10^5, 10^6,10^7\}.$
		\item We compare 5 methods:
		
		\bit
			\item \tcb{S-subGD}: our method
			
			\item \tcb{QR}: ``standard", R package \texttt{quantreg}. 
			
			\item \tcb{CONQUER-plugin}: He et al (2021), R package
			\texttt{conquer}.
			
			\item \tcb{CONQUER-bootstrap}: He et al (2021), R package
			\texttt{conquer}.
			
			\item \tcb{SGD-bootstrap}: Fang et al.~(2018). Bootstrap-online learning
			
			
		\eit
		
		\medskip
		\item Constraints: 10 hour and 192 Gb RAM for one replication.
	\end{itemize}
	
	
\end{frame}



\begin{frame}{Can you compute?} 
	
	\bit
		
		\item \tcb{S-subGD}: Yes for all cases.
		
		\item \tcb{QR}:  out-of memory when $n= 10^7$ and $d=320$.
		
		\item \tcb{Conquer-plugin}:  out-of memory when $n=10^6$.
		
		\item \tcb{Conquer-bootstrap}: out-of-time when $n= 10^7$ and $d=320$.
		
		\item \tcb{SGD-bootstrap}: Survived but 100 times slower than S-subGD.
		
		
	\eit
	
	
	
\end{frame}

\begin{frame}
	\begin{figure}[!htbp]
		\caption{Computation Time} \label{fig:time}
		\centering
		\vskip10pt
		\includegraphics[width=\textwidth]{figures/fig_time.png}
	\end{figure}

Note: Observe 'NA' for several cases.
	
\end{frame}
\begin{frame}
	\begin{figure}[!htbp]
		\caption{Relative Computation Time} \label{fig:time}
		\centering
		\vskip10pt
		\includegraphics[width=\textwidth]{figures/fig_rel_time.png}
	\end{figure}
	
\end{frame}

\begin{frame}
	
	\begin{figure}[!htbp]
		\caption{Coverage Rate} \label{fig:ci}
		\centering
		\vskip10pt
		\includegraphics[width=\textwidth]{figures/fig_coverage.png}
	\end{figure}
	
\end{frame}


\begin{frame}
	
	\begin{figure}[!htbp]
		\caption{Confidence Interval Length} \label{fig:ci}
		\centering
		\vskip10pt
			\includegraphics[width=\textwidth]{figures/fig_ci_length.png}
	\end{figure}
	
\end{frame}


\begin{frame}
\frametitle{Stretch Out}
	\begin{table}[htb]
		\centering
		\caption{Performance of S-subGD: $n=10^7$} 
		\label{tb_SGD_rs}
		\begin{tabular}{cccc
%				S[table-format=3.2]S[table-format=1.3]S[table-format=1.4]
			}
			\hline
			{$d$} & {Time (sec.)} & {Coverage Rate} & {CI Length} \\ 
			\hline
			10 & 5.87 & 0.965 & 0.0020 \\ 
			20 & 11.05 & 0.955 & 0.0020 \\ 
			40 & 21.86 & 0.954 & 0.0020 \\ 
			80 & 43.12 & 0.952 & 0.0020 \\ 
			160 & 81.35 & 0.953 & 0.0021 \\ 
			320 & 166.40 & 0.963 & 0.0011 \\ 
			1000 & 762.16 & 0.925 & 0.0461 \\ 
			\hline
		\end{tabular}
	\end{table}
	
	
\end{frame}


%\section{Linear and Logistic Regression}
%
%\begin{frame}
%\vfill
%\centering
%\LARGE{Numerical Experiments}
%\vfill
%\end{frame}
%
%
%\begin{frame}[allowframebreaks]{Simulations: Linear Regression}
%
%
%
%The data are generated from
%\begin{align*}
%y_{t} = x_t'\beta^* + \varepsilon_t~~\mbox{for}~~t=1,\ldots,n,
%\end{align*}
%where $x_t$ is a $d$-dimensional covariate vector generated from the multivariate normal distribution $\mathcal{N} (0,I_d)$, $\varepsilon_t$ is from $N(0,1)$, and $\beta^*$ is equi-spaced on the interval $[0,1]$.
%
%%\medskip
%%This experimental design is the same as that of Zhu et al.~(2021).
%
%\medskip
%The dimension of $x$ is set to $d=5,20$.
%
%\medskip
%We consider different combination of the learning rate $\gamma_{t}=\gamma_0 t^{-a}$ by setting $\gamma_0=0.5, 1$ and $a = 0.505, 0.667$.
%
%\medskip
%The sample size set to be $n=100,000$.
%The initial value $\beta_0$ is set to be zero.
%
%\medskip
%In case of $d=20$, we burn in around 1\% of observations and start to estimate $\bar{\beta}_t$ from $t=1000$.
%Finally, the simulation results are based on $1000$ replications.
%
%\medskip
%We compare the performance of the proposed random scaling method with the state-of-the-art methods in the literature, especially the plug-in method in Chen et al. (2020) and the recursive batch-mean method in Zhu et al.~(2021).%\footnote{We are grateful to the authors of Zhu et al. (2021) for providing us their code.}
%
%\medskip
% The performance is measured by three statistics: the coverage rate, the average length of the 95\% confidence interval, and the average computation time.
%Note that the nominal coverage probability is set at 0.95.
%%Figure~\ref{fig:main-coef} shows the coverage rates and the lengths of the 95\% confidence interval for each coefficient when $d=5,\gamma_0=0.5,$ and $\alpha=0.505$, which was the step size used in  Zhu et al. (2021).
%%When we restrict to our attention to a specific method, we do not see much difference across different coefficients.
%
%\medskip
%For brevity, we focus on the first coefficient $\beta_1$ hereafter. The results are similar across different coefficients.
%
%
%\begin{figure}[htp]
%\tiny
%\caption{Linear Regression: $d \in \{5, 20\}$, $\gamma_0=0.5$, $a= 0.505$  for $\gamma_t=\gamma_0 t^{-a}$ as in Zhu et al. (2021)}\label{fig:m03-maintext}
%\centering
%\begin{tabular}{c}
%%m03
%\multicolumn{1}{c}{\underline{$d=5$}}\\
%\includegraphics[scale=0.12]{figures/fig-coverage-d5-m01.png}
%\includegraphics[scale=0.12]{figures/fig-length-d5-m01.png}
%\includegraphics[scale=0.12]{figures/fig-time-d5-m01.png}\\
%\\
%\multicolumn{1}{c}{\underline{$d=20$}}\\
%\includegraphics[scale=0.12]{figures/fig-coverage-d20-m01.png}
%\includegraphics[scale=0.12]{figures/fig-length-d20-m01.png}
%\includegraphics[scale=0.12]{figures/fig-time-d20-m01.png}\\
%\end{tabular}
%\end{figure}
%
%
%
%
%\begin{figure}[hpbt]
%\tiny
%\caption{Linear Regression: $d = 5$, $\gamma_0\in\{0.5, 1\}$, $a\in\{0.505, 0.667\}$  for $\gamma_t=\gamma_0 t^{-a}$
%}\label{fig:compare-lr-maintext}
%\centering
%\begin{tabular}{cc}
%%m03
%%\multicolumn{2}{c}{\underline{$d=5$}}\\
%{\underline{$\gamma_0=0.5, a=0.505$}} & {\underline{$\gamma_0=0.5, a=0.667$}} \\
%\includegraphics[scale=0.12]{figures/fig-coverage-d5-m01.png} & \includegraphics[scale=0.12]{figures/fig-coverage-d5-m02.png} \\
%{\underline{$\gamma_0=1, a=0.505$}} & {\underline{$\gamma_0=1, a=0.667$}}\\
% \includegraphics[scale=0.12]{figures/fig-coverage-d5-m03.png} & \includegraphics[scale=0.12]{figures/fig-coverage-d5-m04.png}\\
%\\
%%\multicolumn{2}{c}{\underline{$d=20$}}\\
%%{\underline{$\gamma_0=0.5, \alpha=0.505$}} & {\underline{$\gamma_0=0.5, \alpha=0.667$}}\\
%%\includegraphics[scale=0.22]{figures/fig-coverage-d20-m03.png} & \includegraphics[scale=0.22]{figures/fig-coverage-d20-m04.png}\\
%%{\underline{$\gamma_0=1, \alpha=0.505$}} & {\underline{$\gamma_0=1, \alpha=0.667$}}\\
%%\includegraphics[scale=0.22]{figures/fig-coverage-d20-m01.png} & \includegraphics[scale=0.22]{figures/fig-coverage-d20-m02.png}\\
%\end{tabular}
%\end{figure}
%
%\end{frame}
%
%
%
%
%
%\begin{frame}[allowframebreaks]{Simulations: Logistic Regression}
%
%
%
%We consider the following logistic regression model:
%\begin{align*}
%y_t = 1(x_t'\beta^* - \varepsilon_t \ge 0)~~\mbox{for}~~t=1,\ldots,n,
%\end{align*}
%where $\varepsilon_t$ follows the standard logistic distribution and $1(\cdot)$ is the indicator function.
%
%\medskip
%We consider a large dimension of \(x_t\) (\(d=200\)) as well as \(d=5, 20\).
%
%\medskip
%All other settings are the same as the linear model.
%
%\framebreak
%
%\begin{table}[htbp]
%\scriptsize
%\centering
%\caption{Logistic Regression, $n=10^5$, $\gamma_0=0.5$, $a=0.505$  for $\gamma_t=\gamma_0 t^{-a}$} \label{tb:logit}
%\begin{tabular}{lccc}
%\hline
%                        &    $d=5$     &     $d=20$     &   $d=200$        \\
%\hline
% \underline{Random Scale}      \\
%Coverage               &    0.930     &     0.929      &     0.919        \\
%Length                 &    0.036     &     0.043      &     0.066   \\
%Time (sec.)            &     8.4      &      11.4      &     170.3     \\
%\hline
%\underline{Batch-mean}  \\
%Coverage               &    0.824 &     0.772      &     0.644        \\
%Length                 &    0.022     &     0.024      &     0.027       \\
%Time (sec.)            &     6.0      &       7.0      &      10.7       \\
%\hline
%\underline{Plug-in}       \\
%Coverage              &    0.953     &     0.946      &     0.944        \\
%Length                 &    0.029     &     0.035      &     0.053        \\
%Time (sec.)            &    55.2      &      66.8      &     955.0       \\
%\hline
%
%\end{tabular}
%\end{table}
%
%%\framebreak
%
%\medskip
%Overall, the simulation results are similar to those in linear regression.
%Random Scale requires more computation time than Batch-mean but is still much faster than Plug-in.
%
%\framebreak
%
%\medskip
%The computation time of Random Scale can be substantially reduced when we are interested in the inference of a single parameter. In such a case, we need to update only a single element of $\hat{V}$ rather than the whole $d\times d$ matrix.
%
%\medskip
%Random Scale can be easily scaled up to $d=800$ with only 11.7 seconds computation time when we are interested in the inference of a single parameter.
%
%%\item The performance of Random-scale is less sensitive to the choice of tuning parameters than Batch-mean.
%
%
%\begin{table}[htbp]
%\centering
%\caption{Logistic Regression: Random Scale Updating a Single Element of $\hat{V}$, $n=10^5$, $\gamma_0=0.5$, $a=0.505$  for $\gamma_t=\gamma_0 t^{-a}$} \label{tb:logit_single}
%\begin{tabular}{lccccc}
%\hline
%& $d=5$ & $d=20$ & $d=200$ & $d=500$ & $d=800$\\
%\hline
%Coverage &  0.930 & 0.929 & 0.919 & 0.927 & 0.931\\
%Length  & 0.037     & 0.043 &   0.066 & 0.133 & 0.196\\
%Time (sec.) &   5.0 &   5.3 &   6.7 &   9.7 &   11.7\\
%\hline
%\end{tabular}
%\end{table}
%
%\end{frame}
%
%
%\begin{frame}[allowframebreaks]{Simulations: Quantile Regression}
%\begin{center}
%\includegraphics[scale=0.27]{figures/Fig_qr_time.png}
%\includegraphics[scale=0.27]{figures/Fig_qr_rel_time.png}
%\includegraphics[scale=0.27]{figures/fig_qe_coverage}
%\includegraphics[scale=0.27]{figures/fig_qr_ci_length}
%\end{center}
%
%
%\end{frame}
%
%
%\begin{frame}{Application: College Premiums}
%We estimate median college premiums for male and female workers based on Hubbard (2011):
%\eqs{
%  \ln(wage_i) = \beta_0 + \beta_1 FEM_i + \beta_2 EDU_i + \beta_3 FEM_i \cdot EDU_i + \theta'X_i + \varepsilon_i,
%}where 
%\bit
%    \item $FEM_i = 1$ if Female
%    \item $EDU_i = 1$ if Some college or above
%    \item $X_i: $ $age$, $age^2$, 8 regional dummies, all interactions with $FEM_i$
%\eit
%We have 23 regressors in total.
%\medskip
%
%Data: IPUMS USA 1990, 5\% extract of the census. $n=3.6$ mil.
%
%%\begin{tiny}
%%
%%\begin{table}[ht]
%%\centering
%%\caption{Computation Time (minutes) } \label{tb:app_computation_time}
%%\begin{tabular}{crrrrrrrrrr}
%%  \hline
%% Sample Size& 10\% & 20\% & 30\% & 40\% & 50\% & 60\% & 70\% & 80\% & 90\% & 100\% \\ 
%%  \hline
%%SGD & 0.4 & 0.7 & 1.1 & 1.5 & 1.9 & 2.3 & 2.7 & 3.0 & 3.5 & 3.8 \\ 
%%QR  & 3.6 & 14.3 & 32.8 & 65.5 & 105.4 & 146.1 & 222.3 & 290.0 & 348.3 & 464.0 \\ 
%%   \hline
%%\end{tabular}
%%\end{table}
%%\end{tiny}
%
%\end{frame} 
%
\section{Application}

\begin{frame}
	\vfill
	\centering
	\LARGE{Application: College Wage Premium}
	\vfill
\end{frame}

\begin{frame}{Gender Gap in College Wage Premium}
	
	\begin{itemize}
		\item The higher college wage premium for women has been suggested as a major cause for attracting more women to colleges than men (e.g., Goldin et al. (2006); Chiappori et al. (2009)).
		
		\item Top coded wage issue: Hubbard (2011) estimated the model by censored regression and quantile regression
		
		\item Our Goals:
		
		\begin{enumerate}
			\item To identify (if any) heterogeneous effects across quantiles.

			\item To understand the difference in gender trends in the college wage premium.

			\item To utilize a large number of controls (census data).
			
			%\item understand the trends in the college wage premium respectively for female and male
			
			
			
		\end{enumerate}
		
	\end{itemize}


\end{frame}

\begin{frame}
\frametitle{Samples from IPUMS}
	\bit
		\item We use the samples over several different years (1980, 1990, 2000-2015) from IPUMS USA.
		
		\item
		In the years from 1980 to 2000, we use the 5\% State sample which is a 1-in-20 national random sample of the population. 
		In the remaining years, we use the American Community Survey (ACS) each year. 
		
		\item The sampling ratio varies from 1-in-261 to 1-in-232 in 2001-2004, but it is set to a 1-in-100 national random sample of the population after 2005. 
		
		\item
		To balance the sample size, we bunch the sample every 5 year after 2001. 		
	\eit
\end{frame}

\begin{frame}
\frametitle{Summary Statistics}
	\begin{table}[ht]
		\centering
		\footnotesize
		\begin{tabular}{crcccc}
			\hline
			Year & Sample Size & $\mathbb E(F$) & $\mathbb E(Educ$) & $\mathbb E(Educ|M$) & $\mathbb E(Edu|F$) \\ 
			\hline
			1980      & 3,659,684 & 0.390 & 0.433 & 0.444 & 0.416 \\ 
			1990      & 4,192,119 & 0.425 & 0.543 & 0.537 & 0.550 \\ 
			2000      & 4,479,724 & 0.439 & 0.600 & 0.578 & 0.629 \\ 
			2001-2005 & 2,493,787 & 0.447 & 0.642 & 0.619 & 0.670 \\ 
			2006-2010 & 4,708,119 & 0.447 & 0.663 & 0.631 & 0.701 \\ 
			2011-2015 & 4,542,874 & 0.447 & 0.686 & 0.646 & 0.735 \\ 
			\hline
		\end{tabular}
	\end{table}
	\bit
		\item  $Educ$ denotes an education dummy for some college or above. 

		\item $White$, $18 \le Age \le 65$,  and Full time workers (30 hours per week)

		\item $Wage \ge \$62$, which is a half of minimum wage earnings in 1980 ($\$3.10 \times 40 \mbox{hours} \times 1/2$). 
		
		\item
		The data cleaning leaves us with 3.6-4.7 million observations besides 2001-2005, where we have around 2.5 million observations.
		
	\eit
	
			\note{The table confirms that female college education has increased substantially over the years and female workers received more college education than male workers since 1990. }

\end{frame}

\begin{frame}
\frametitle{Regression Model}
\begin{footnotesize}
\begin{align*}
    \log(Wage_i) = & \beta_0 + \beta_1 Female_i + \beta_2 Educ_i + \beta_3 Female_i\cdot Educ_i\\ & + \theta_1'X_i + \theta_2' (X_i\cdot Female_i) + \varepsilon_i,
\end{align*}
\end{footnotesize}
	\bit
		\item For control variable $X_i$, we use 12 age group dummies with a four-year interval, 51 states dummies (including D.C.), and their interactions. Note that $(X_i\cdot Female_i)$ implies that there exist up to 3-way interactions. 

		\item The model contains 1226 covariates in total. 

		\item We also add 4 additional year dummies for the 5-year combined samples after 2001. 

		
	\eit

\end{frame}



\begin{frame}
	
	\bigskip
	
	\begin{figure}[!h]
		\centering
		\caption{College Wage Premium: %Model 3 with 
			Combining 5-Year Data %. Full-time
		}
	\bigskip
			\includegraphics[width=0.3\textwidth]{figures/chart_1_03_FullT_5yr.png}  \includegraphics[width=0.3\textwidth]{figures/chart_5_03_FullT_5yr.png} \includegraphics[width=0.3\textwidth]{figures/chart_9_03_FullT_5yr.png} 
			
	\end{figure}
	
\end{frame}

\begin{frame}

\begin{figure}
    \centering
    \caption{Difference of College Premium: Combining 5-Year Data} %Model 3 with combining 5-year data. Full-time}
    \label{fig:diff_5yr_full}
    \hskip15pt
    \begin{tabular}{c c c}
        \includegraphics[width=0.3\textwidth]{figures/diff_chart_0.1_03_FullT_5yr.png} & \includegraphics[width=0.3\textwidth]{figures/diff_chart_0.5_03_FullT_5yr.png} & \includegraphics[width=0.3\textwidth]{figures/diff_chart_0.9_03_FullT_5yr.png} 
    \end{tabular}
  %  \flushleft{\footnotesize Notes. We estimate the model with combining 5 years after 2001.}
\end{figure}
\end{frame}

\begin{frame}
	
	\begin{table}[htb]
		\centering
		\footnotesize
		\caption{College Wage Premium: $\tau=0.5$}\label{tb:median_5yr_full}
		\begin{tabular}{ccccc}
			\hline
			Year & Female & Male & Difference & Time (min.) \\ 
			\hline
			\underline{$\tau = 0.5$}\\
			1980 & 0.2365 & 0.1988 & 0.0377 & 29.4 \\ 
			& [0.2294,0.2435] & [0.1945,0.2030] & [0.0291,0.0463] &  \\ 
			1990 & 0.3667 & 0.2962 & 0.0705 & 34.2 \\ 
			& [0.3603,0.3732] & [0.2942,0.2982] & [0.0634,0.0777] &  \\ 
			2000 & 0.4101 & 0.3439 & 0.0662 & 36.7 \\ 
			& [0.4056,0.4146] & [0.3372,0.3506] & [0.0552,0.0772] &  \\ 
			2001-2005 & 0.4468 & 0.3854 & 0.0613 & 20.2 \\ 
			& [0.4369,0.4567] & [0.3765,0.3944] & [0.0554,0.0673] &  \\ 
			2006-2010 & 0.4791 & 0.4271 & 0.0520 & 47.7 \\ 
			& [0.4748,0.4834] & [0.4174,0.4368] & [0.0454,0.0585] &  \\ 
			2011-2015 & 0.4957 & 0.4498 & 0.0458 & 46.0 \\ 
			& [0.4887,0.5027] & [0.4455,0.4542] & [0.0348,0.0568] &  \\ 
			\hline
		\end{tabular}

	\end{table}
	
	
\end{frame}

\section{Conclusion}

\begin{frame}{Conclusion} 
	\begin{itemize}
		
		\item We provide a new scalable on-line inference method for quantile regression with ``ultra-large" sample sizes and a large number of covariates. 
		
		\item It is efficient in terms of both computation time and the memory usage. 

		\item Simulations and the empirical application shows that the S-subGD method could open a new realm of empirical studies. 
		
		\item We are currently working on:
		
		\begin{itemize}
			
			\item Endogenous regressor (GMM or Structural models)

			\item Cluster robust inference (federated learning)
			
			\item High-dimensional settings
			
			
		\end{itemize}
		
		
	\end{itemize} 
\end{frame}

\fm{
\vskip30pt
\centering
   \includegraphics[scale=1]{figures/thankyou.jpg}
}

\end{document}


